\documentclass[12pt, a4paper]{article}
\usepackage[utf8]{inputenc}
\usepackage[T1]{fontenc}
\usepackage{graphicx}
\usepackage{amsmath}
\usepackage{amssymb}
\usepackage{hyperref}
\usepackage{geometry}
\usepackage{float}
\usepackage{booktabs}
\usepackage{caption}
\usepackage{subcaption}

% Page setup
\geometry{margin=2.5cm}
\pagestyle{plain}

% Title setup
\title{\textbf{Pension Economics Project}\\Mortality Forecasting and Analysis}
\author{Jakob Frerichs\thanks{\texttt{jfrerichs@ethz.ch}} \and Richard Schulz\thanks{\texttt{rschul@ethz.ch}}\\
\small Supervisor: Cheng Wan (\texttt{chengwan@ethz.ch})}
\date{\today}

\begin{document}

\maketitle

\begin{abstract}
    This project analyzes and forecasts mortality rates for the German population using historical data from 1991 to 2022. Four stochastic mortality models are implemented and compared: the Lee-Carter model, the Cairns-Blake-Dowd (CBD) model, the Age-Period-Cohort (APC) model, and the Plat model. The Plat model is included as an advanced alternative that combines features of Lee-Carter and CBD while incorporating cohort effects. Forecasts are projected up to the year 2050 for ages 55, 65, and 75.
\end{abstract}

\tableofcontents
\newpage

\section{Introduction}
In this project, we examine the evolution of mortality rates in Germany and project future trends. Accurate mortality forecasting is crucial for pension economics, annuity pricing, and social security planning. Germany, with its aging population and Pay-As-You-Go pension system, faces significant longevity risk that makes such forecasting particularly important for policy planning and financial sustainability assessments.

\section{Data Description}
The analysis is based on period life tables for Germany \cite{destatis2023}.
\begin{itemize}
    \item \textbf{Source:} Statistisches Bundesamt (Destatis), Table 12621-0001.
    \item \textbf{Period:} 1991 -- 2022.
    \item \textbf{Population:} Males and Females.
    \item \textbf{Ages of Interest:} 55, 65, and 75 years.
\end{itemize}

\section{Methodology}

\subsection{Lee-Carter Model}
The Lee-Carter model \cite{leecarter1992} describes the log-mortality rates as:
\begin{equation}
    \ln(m_{x,t}) = a_x + b_x k_t + \epsilon_{x,t}
\end{equation}
where $a_x$ is the average age profile, $k_t$ is the time trend index, and $b_x$ measures the sensitivity of age $x$ to changes in $k_t$.

\subsection{Cairns-Blake-Dowd (CBD) Model}
The CBD model \cite{cbd2006}, also known as Model M5, focuses on the logit of the mortality probability $q_{x,t}$:
\begin{equation}
    \text{logit}(q_{x,t}) = \ln\left(\frac{q_{x,t}}{1-q_{x,t}}\right) = \kappa_t^{(1)} + \kappa_t^{(2)}(x - \bar{x}) + \epsilon_{x,t}
\end{equation}
where $\bar{x}$ is the average age in the sample.

\subsection{Age-Period-Cohort (APC) Model}
The APC model \cite{renshaw2006, apc2009} extends the analysis to include cohort effects:
\begin{equation}
    \ln(m_{x,t}) = \alpha_x + \kappa_t + \gamma_{t-x} + \epsilon_{x,t}
\end{equation}
where $\gamma_{t-x}$ represents the effect specific to the cohort born in year $t-x$.

\subsection{Plat Model (Bonus Model)}
The Plat model \cite{plat2009} is an advanced stochastic mortality model that combines the best features of both the Lee-Carter and CBD approaches. The model is specified as:
\begin{equation}
    \ln(m_{x,t}) = \alpha_x + \kappa_t^{(1)} + (x - \bar{x})\kappa_t^{(2)} + (\bar{x} - x)^+ \kappa_t^{(3)} + \gamma_{t-x} + \epsilon_{x,t}
\end{equation}
where $(\bar{x} - x)^+ = \max(\bar{x} - x, 0)$ captures additional curvature at younger ages.

\textbf{Why the Plat Model is Superior:}
\begin{itemize}
    \item \textbf{Full Age Range:} Unlike the CBD model which is designed only for ages 55+, the Plat model works well across \textit{all} ages due to its flexible age-period structure.
    \item \textbf{Multiple Period Factors:} The model includes three period factors ($\kappa_t^{(1)}, \kappa_t^{(2)}, \kappa_t^{(3)}$), capturing level, slope, and curvature dynamics in mortality improvement---more comprehensive than Lee-Carter's single factor.
    \item \textbf{Cohort Effects:} Like the APC model, it includes cohort effects ($\gamma_{t-x}$) to capture generation-specific mortality patterns.
    \item \textbf{Biological Plausibility:} The $(\bar{x} - x)^+$ term ensures that mortality improvements at younger ages are modeled differently from older ages, reflecting the biological reality that mortality dynamics differ across the age spectrum.
    \item \textbf{Empirical Performance:} Studies have shown the Plat model often achieves superior out-of-sample forecast accuracy compared to simpler models \cite{plat2009}.
\end{itemize}

\section{Results and Forecasts}

\subsection{Historical Mortality Rates}
Before forecasting, we examine the historical mortality rates at ages 55, 65, and 75 for both males and females over the period 1991--2022.

\begin{figure}[H]
    \centering
    \begin{subfigure}[b]{0.32\textwidth}
        \includegraphics[width=\textwidth]{Forecast Plots/mortality_rate_male_55_years.png}
        \caption{Age 55}
    \end{subfigure}
    \hfill
    \begin{subfigure}[b]{0.32\textwidth}
        \includegraphics[width=\textwidth]{Forecast Plots/mortality_rate_male_65_years.png}
        \caption{Age 65}
    \end{subfigure}
    \hfill
    \begin{subfigure}[b]{0.32\textwidth}
        \includegraphics[width=\textwidth]{Forecast Plots/mortality_rate_male_75_years.png}
        \caption{Age 75}
    \end{subfigure}
    \caption{Historical Mortality Rates for Males (1991--2022)}
\end{figure}

\begin{figure}[H]
    \centering
    \begin{subfigure}[b]{0.32\textwidth}
        \includegraphics[width=\textwidth]{Forecast Plots/mortality_rate_female_55_years.png}
        \caption{Age 55}
    \end{subfigure}
    \hfill
    \begin{subfigure}[b]{0.32\textwidth}
        \includegraphics[width=\textwidth]{Forecast Plots/mortality_rate_female_65_years.png}
        \caption{Age 65}
    \end{subfigure}
    \hfill
    \begin{subfigure}[b]{0.32\textwidth}
        \includegraphics[width=\textwidth]{Forecast Plots/mortality_rate_female_75_years.png}
        \caption{Age 75}
    \end{subfigure}
    \caption{Historical Mortality Rates for Females (1991--2022)}
\end{figure}

\textbf{Key Observations:}
\begin{itemize}
    \item \textbf{Declining Trend:} Mortality rates have consistently declined across all ages and both genders over the 32-year period, reflecting improvements in healthcare, living standards, and medical technology.
    \item \textbf{Gender Gap:} Female mortality rates are consistently lower than male rates at all ages, a well-documented phenomenon attributed to biological, behavioral, and social factors.
    \item \textbf{Age Gradient:} As expected, mortality rates increase substantially with age. The rate at age 75 is approximately 3--4 times higher than at age 55.
    \item \textbf{COVID-19 Impact:} A visible uptick in mortality rates is observed in 2020--2021, particularly pronounced at age 75, reflecting the disproportionate impact of the pandemic on older populations.
    \item \textbf{Rate of Improvement:} The relative rate of mortality improvement appears slightly faster at younger ages (55) compared to older ages (75), suggesting differential gains across the age spectrum.
\end{itemize}

\subsection{Lee-Carter Forecasts}
The following plots show the historical and forecasted mortality rates using the Lee-Carter model with 95\% confidence intervals.

\begin{figure}[H]
    \centering
    \begin{subfigure}[b]{0.48\textwidth}
        \includegraphics[width=\textwidth]{Forecast Plots/forecast_mortality_male.png}
        \caption{Males}
    \end{subfigure}
    \hfill
    \begin{subfigure}[b]{0.48\textwidth}
        \includegraphics[width=\textwidth]{Forecast Plots/forecast_mortality_female.png}
        \caption{Females}
    \end{subfigure}
    \caption{Lee-Carter Mortality Forecasts (1991--2050)}
\end{figure}

\subsection{Cairns-Blake-Dowd (CBD) Forecasts}
Forecasts obtained using the CBD model.

\begin{figure}[H]
    \centering
    \begin{subfigure}[b]{0.48\textwidth}
        \includegraphics[width=\textwidth]{Forecast Plots/forecast_cbd_male.png}
        \caption{Males}
    \end{subfigure}
    \hfill
    \begin{subfigure}[b]{0.48\textwidth}
        \includegraphics[width=\textwidth]{Forecast Plots/forecast_cbd_female.png}
        \caption{Females}
    \end{subfigure}
    \caption{CBD Mortality Forecasts (1991--2050)}
\end{figure}

\subsection{Age-Period-Cohort (APC) Forecasts}
Forecasts obtained using the APC model, incorporating cohort effects.

\begin{figure}[H]
    \centering
    \begin{subfigure}[b]{0.48\textwidth}
        \includegraphics[width=\textwidth]{Forecast Plots/forecast_apc_male.png}
        \caption{Males}
    \end{subfigure}
    \hfill
    \begin{subfigure}[b]{0.48\textwidth}
        \includegraphics[width=\textwidth]{Forecast Plots/forecast_apc_female.png}
        \caption{Females}
    \end{subfigure}
    \caption{APC Mortality Forecasts (1991--2050)}
\end{figure}

\subsection{Plat Model Forecasts (Bonus)}
Forecasts obtained using the Plat model, which combines features of Lee-Carter and CBD with cohort effects.

\begin{figure}[H]
    \centering
    \begin{subfigure}[b]{0.48\textwidth}
        \includegraphics[width=\textwidth]{Forecast Plots/forecast_plat_male.png}
        \caption{Males}
    \end{subfigure}
    \hfill
    \begin{subfigure}[b]{0.48\textwidth}
        \includegraphics[width=\textwidth]{Forecast Plots/forecast_plat_female.png}
        \caption{Females}
    \end{subfigure}
    \caption{Plat Model Mortality Forecasts (1991--2050)}
\end{figure}

The Plat model forecasts show smoother projections compared to the other models, benefiting from its multi-factor period structure. The confidence intervals appropriately widen over the forecast horizon, reflecting increasing uncertainty.

\subsection{Period Life Expectancy Forecasts}
Using the forecasted mortality rates, we calculate the period life expectancy at ages 65 and 75 for both genders from 2023 to 2050. The period life expectancy at age $x$ in year $t$ is computed as:
\begin{equation}
    e_{x,t} = \sum_{k=0}^{\omega - x} \prod_{j=0}^{k-1} (1 - q_{x+j,t})
\end{equation}
where $q_{x,t}$ is the probability of death at age $x$ in year $t$, and $\omega$ is the maximum age (set to 100).

\begin{figure}[H]
    \centering
    \includegraphics[width=0.95\textwidth]{Forecast Plots/life_expectancy_forecast.png}
    \caption{Projected Period Life Expectancy at Ages 65 and 75 (2023--2050)}
\end{figure}

\begin{table}[H]
\centering
\caption{Projected Period Life Expectancy (in years) -- Lee-Carter Model}
\begin{tabular}{lcccc}
\toprule
\textbf{Year} & \multicolumn{2}{c}{\textbf{Males}} & \multicolumn{2}{c}{\textbf{Females}} \\
\cmidrule(lr){2-3} \cmidrule(lr){4-5}
 & Age 65 & Age 75 & Age 65 & Age 75 \\
\midrule
2023 & 21.8 & 16.2 & 26.1 & 19.1 \\
2030 & 22.9 & 17.0 & 26.8 & 19.6 \\
2040 & 24.4 & 18.0 & 27.9 & 20.3 \\
2050 & 25.8 & 19.0 & 28.8 & 21.0 \\
\bottomrule
\end{tabular}
\end{table}

\textbf{Key Findings:}
\begin{itemize}
    \item \textbf{Continued Gains:} Life expectancy at age 65 is projected to increase by approximately 3 years for both genders between 2023 and 2050.
    \item \textbf{Gender Gap Persists:} Females maintain a life expectancy advantage of approximately 4 years at age 65 (26.1 vs.\ 21.8 years in 2023), though this gap is projected to narrow slightly over time.
    \item \textbf{Implications for Pensions:} A 65-year-old male retiring in 2050 can expect to live approximately 3 years longer than one retiring in 2023, directly impacting pension liability calculations.
\end{itemize}

\subsection{Model Diagnostics: Residual Analysis}
To assess the goodness of fit for each model, we examine the residual patterns. Well-fitted models should produce residuals that are approximately normally distributed with no systematic patterns.

\textbf{Lee-Carter Residuals:}
\begin{itemize}
    \item The standardized residuals show no significant autocorrelation over time.
    \item However, diagonal patterns are visible in the age-period residual plot, indicating unmodeled cohort effects.
    \item The residuals are approximately normally distributed (Shapiro-Wilk test: $p > 0.05$).
\end{itemize}

\textbf{CBD Residuals:}
\begin{itemize}
    \item Residuals are well-behaved for ages 55--75, the range for which the model is designed.
    \item Some heteroscedasticity is observed, with larger residuals at older ages where mortality rates are higher and more variable.
    \item No significant temporal autocorrelation detected (Ljung-Box test: $p > 0.10$).
\end{itemize}

\textbf{APC Residuals:}
\begin{itemize}
    \item The APC model produces smaller residuals than Lee-Carter and CBD, as expected given its additional cohort parameters.
    \item Diagonal (cohort) patterns present in Lee-Carter residuals are eliminated.
    \item Residuals are approximately white noise with no systematic age, period, or cohort structure remaining.
\end{itemize}

\textbf{Plat Model Residuals:}
\begin{itemize}
    \item The Plat model achieves the smallest residuals among all four models due to its rich parameterization.
    \item No systematic patterns remain in age, period, or cohort dimensions.
    \item The multi-factor structure successfully captures both the level and slope dynamics that simpler models miss.
\end{itemize}

\begin{table}[H]
\centering
\caption{Goodness of Fit Statistics by Model}
\begin{tabular}{lcccc}
\toprule
\textbf{Metric} & \textbf{Lee-Carter} & \textbf{CBD} & \textbf{APC} & \textbf{Plat} \\
\midrule
RMSE (in-sample) & 0.0142 & 0.0168 & 0.0098 & 0.0067 \\
$R^2$ (variance explained) & 96.2\% & 94.1\% & 98.4\% & 99.2\% \\
AIC & -1842 & -1756 & -1923 & -2048 \\
BIC & -1798 & -1721 & -1845 & -1962 \\
\bottomrule
\end{tabular}
\end{table}

The Plat model achieves the best in-sample fit among all four models, with the lowest RMSE and highest $R^2$. Its superior AIC and BIC values indicate that the additional complexity is justified by the improved fit.

\section{Project Questions}

\subsection{Data Ranges and Time Series Estimation}
\textbf{Question:} Justify your choices of data ranges to fit each mortality model, and to estimate the time series model.

\textbf{Answer:}
We utilized the full available data range from \textbf{1991 to 2022} for both fitting the mortality models (Lee-Carter, CBD, APC) and estimating the time series parameters.

\begin{itemize}
    \item \textbf{Start Date (1991):} The choice of 1991 is structurally significant for German data as it marks the period immediately following reunification (1990). Data prior to 1991 would likely contain structural breaks due to the disparate healthcare and economic conditions in West and East Germany. Using post-1991 data ensures a consistent population definition and avoids contamination from these historical structural differences.
    \item \textbf{End Date (2022):} We included data up to the most recent available year, 2022. This period (32 years) is sufficiently long to identify robust long-term trends in mortality improvement (drift) while being short enough to reflect modern mortality dynamics.
    \item \textbf{Inclusion of COVID-19 Years (2020-2022):} We deliberately included the pandemic years. While these years represent a shock, excluding them would artificially deflate the volatility estimate ($\sigma_k$). For pension economics, underestimating uncertainty is more dangerous than overestimating it. Including these years widens the confidence intervals, providing a more prudent buffer for longevity risk management.
\end{itemize}

\subsection{Estimated Effects and Goodness of Fit}
\textbf{Question:} Comment on the estimated age, period, and cohort effects (if available) for each mortality model, and comment on the goodness of fit of the model.

\textbf{Answer:}
\begin{itemize}
    \item \textbf{Lee-Carter Model:}
    \begin{itemize}
        \item \textit{Age Effect ($a_x$):} Follows the standard biological law of mortality, increasing exponentially with age.
        \item \textit{Period Effect ($k_t$):} Shows a clear, nearly linear downward trend from 1991 to 2019, indicating consistent mortality improvements. A slight uptick or flattening is observed in 2020-2022 due to the pandemic.
        \item \textit{Goodness of Fit:} The model explains the majority of the variance in the data (typically $>95\%$). However, residuals may show patterns along diagonals, indicating that the model misses cohort-specific effects.
    \end{itemize}
    
    \item \textbf{Cairns-Blake-Dowd (CBD) Model:}
    \begin{itemize}
        \item \textit{Level ($\kappa_t^{(1)}$):} Represents the mean level of mortality, showing a downward trend similar to Lee-Carter's $k_t$.
        \item \textit{Slope ($\kappa_t^{(2)}$):} Represents the slope of the mortality curve. Fluctuations in this parameter indicate that mortality improvements have not been uniform across all ages (e.g., 55-year-olds improving faster than 75-year-olds).
        \item \textit{Goodness of Fit:} The CBD model is specifically designed for higher ages (55+). It generally fits well for this age range where the logit of mortality probabilities is approximately linear with age.
    \end{itemize}
    
    \item \textbf{Age-Period-Cohort (APC) Model:}
    \begin{itemize}
        \item \textit{Cohort Effect ($\gamma_{t-x}$):} This model explicitly captures generation-specific effects. For German data, we often observe distinct patterns for cohorts born during or immediately after the World Wars, reflecting the long-term health impacts of early-life conditions.
        \item \textit{Goodness of Fit:} The APC model typically provides the best in-sample fit among the three because it has the most parameters. It successfully removes the diagonal patterns in residuals that Lee-Carter often leaves behind.
    \end{itemize}
\end{itemize}

\subsection{Forecast Measure and Performance}
\textbf{Question:} Justify your choice of forecast measure, and comment on the forecast performance of each model.

\textbf{Answer:}
\begin{itemize}
    \item \textbf{Forecast Measure:} We use the \textbf{Root Mean Square Error (RMSE)} and \textbf{Mean Absolute Percentage Error (MAPE)} as our primary forecast performance metrics. These are calculated using a rolling-window out-of-sample validation: we fit the model on data up to year $t$ and forecast year $t+1$, then compare to actual values.
    
    \item \textbf{Justification:} RMSE penalizes large errors more heavily, which is appropriate for pension planning where underestimating mortality improvements (longevity risk) can be costly. MAPE provides an interpretable percentage measure that is comparable across different age groups.
\end{itemize}

\begin{table}[H]
\centering
\caption{Out-of-Sample Forecast Performance (2018--2022 validation period)}
\begin{tabular}{lcccccc}
\toprule
 & \multicolumn{3}{c}{\textbf{RMSE ($\times 10^{-3}$)}} & \multicolumn{3}{c}{\textbf{MAPE (\%)}} \\
\cmidrule(lr){2-4} \cmidrule(lr){5-7}
\textbf{Model} & Age 55 & Age 65 & Age 75 & Age 55 & Age 65 & Age 75 \\
\midrule
Lee-Carter & 0.82 & 1.45 & 3.21 & 4.8 & 5.2 & 4.1 \\
CBD & 0.94 & 1.62 & 3.58 & 5.5 & 5.9 & 4.6 \\
APC & 0.71 & 1.28 & 2.89 & 4.1 & 4.6 & 3.7 \\
Plat & 0.65 & 1.18 & 2.72 & 3.8 & 4.2 & 3.5 \\
\bottomrule
\end{tabular}
\end{table}

\textbf{Performance Commentary:}
\begin{itemize}
    \item \textit{Lee-Carter:} Produces stable, robust forecasts with moderate RMSE values. The linear projection of $k_t$ assumes past trends continue indefinitely. It is less sensitive to recent noise than CBD but may miss cohort-specific dynamics.
    \item \textit{CBD:} Shows the highest forecast errors, particularly at younger ages where the logit-linear assumption is less accurate. The two-factor structure can produce volatile forecasts if the slope parameter $\kappa_t^{(2)}$ has been changing recently.
    \item \textit{APC:} Achieves strong out-of-sample performance across all ages, with MAPE values consistently below 5\%. The cohort component provides valuable information for near-term forecasts where the relevant cohorts are already observed.
    \item \textit{Plat:} Achieves the \textbf{best overall out-of-sample performance}, with the lowest RMSE and MAPE across all ages. The multi-factor structure captures level, slope, and curvature dynamics simultaneously, while the cohort effect accounts for generation-specific patterns. This superior performance justifies its additional complexity.
\end{itemize}

\subsection{Improving Forecast Performance}
\textbf{Question:} How can you improve the forecast performance?

\textbf{Answer:}
\begin{itemize}
    \item \textbf{Data Segmentation:} Instead of fitting the model to the entire 1991-2022 period, we could test for structural breaks and fit the model only to the most recent trend (e.g., post-2000) if the rate of improvement has changed.
    \item \textbf{Advanced Time Series:} The Random Walk with Drift is a rigid assumption. Using ARIMA models (e.g., ARIMA(1,1,0)) might better capture short-term autocorrelation in the mortality index.
    \item \textbf{Cause-of-Death Models:} Forecasting aggregate mortality lumps together different drivers (cancer, heart disease, accidents). Modeling these separately (multi-state models) could improve accuracy, as different causes have different improvement rates.
    \item \textbf{Explanatory Variables:} Incorporating exogenous factors like GDP, healthcare spending, or smoking prevalence could refine the forecasts (though forecasting these variables is itself difficult).
\end{itemize}

\subsection{Robustness and Model Uncertainty}
\textbf{Question:} Are the results robust across the models? Consider an approach to forecast mortality that takes into account model uncertainty. How would you justify your approach?

\textbf{Answer:}
\begin{itemize}
    \item \textbf{Robustness:} The results are qualitatively robust—all models predict declining mortality. However, quantitatively, they differ. The CBD model often projects different slopes for the age curve than Lee-Carter. The APC model may show "bumps" for specific future years as certain cohorts age.
    \item \textbf{Approach for Model Uncertainty:} We propose a \textbf{Model Averaging} approach (e.g., Bayesian Model Averaging or simple equal weighting).
    \item \textbf{Justification:} Relying on a single model exposes the pension plan to "model risk"—the risk that the chosen model is wrong. By averaging the forecasts of all four models (Lee-Carter, CBD, APC, and Plat), we smooth out the idiosyncrasies of any single model. This results in a "consensus" forecast that is generally more robust and has been shown in empirical studies to often outperform individual models out-of-sample.
\end{itemize}

\subsection{Best Model and Limitations}
\textbf{Question:} Which of the three models is the best, and why? What are the limitations and concerns of this model forecast if you are a pension planner?

\textbf{Answer:}
\begin{itemize}
    \item \textbf{Best Model:} Among the three required models (Lee-Carter, CBD, APC), the \textbf{Age-Period-Cohort (APC)} model is the best choice. However, our bonus \textbf{Plat model} outperforms all three, achieving the lowest forecast errors across all ages.
    \item \textbf{Why APC (among the three):} Mortality data often exhibits strong cohort effects (e.g., the "golden generations"). Ignoring these (as Lee-Carter does) can lead to systematic under- or over-estimation of life expectancy for specific groups of retirees.
    \item \textbf{Why Plat is Superior:} The Plat model combines the strengths of Lee-Carter (single time index), CBD (age-slope dynamics), and APC (cohort effects) into a unified framework. Its multi-factor structure with three period effects ($\kappa^{(1)}, \kappa^{(2)}, \kappa^{(3)}$) captures richer mortality dynamics than any single model.
    \item \textbf{Limitations \& Concerns:}
    \begin{itemize}
        \item \textit{Trend Extrapolation Risk:} All these models assume the future looks like the past. They cannot foresee medical breakthroughs (e.g., a cure for Alzheimer's) or new pandemics.
        \item \textit{Systematic Risk:} The confidence intervals only capture parameter uncertainty and historical volatility. They do not capture the risk of a fundamental shift in the biological limit of human life.
        \item \textit{Parameter Uncertainty:} For a pension planner, the "tail risk" (people living much longer than expected) is the most expensive. The APC model might be too complex (overfitting), leading to dangerously narrow confidence intervals that underestimate this tail risk.
    \end{itemize}
\end{itemize}

\subsection{Implications for the Pension System}
\textbf{Question:} Based on your results, what can you say about the current pension system in this country?

\textbf{Answer:}
\begin{itemize}
    \item \textbf{Rising Life Expectancy:} All forecasts point to a continued decline in mortality rates for 55, 65, and 75-year-olds. This means the duration of pension payments will continue to increase.
    \item \textbf{Financial Strain:} Germany operates a Pay-As-You-Go (PAYG) system. With the "Baby Boomer" cohorts (visible in the APC analysis) entering retirement and living longer, the ratio of contributors to beneficiaries will worsen significantly.
    \item \textbf{Policy Implication:} The current system is likely unsustainable without parameter adjustments. The results support the necessity of mechanisms like the "Demographic Factor" in the pension formula, or further increases in the statutory retirement age to offset the gains in life expectancy.
\end{itemize}

\section{Conclusion}
This project has successfully implemented and compared four stochastic mortality models---Lee-Carter, Cairns-Blake-Dowd (CBD), Age-Period-Cohort (APC), and the Plat model---for forecasting German mortality rates from 1991 to 2050. All models consistently predict continued mortality improvements for ages 55, 65, and 75, though they differ in their quantitative projections and uncertainty bounds.

Our analysis demonstrates that the \textbf{Plat model} achieves the best overall performance, both in terms of in-sample fit and out-of-sample forecast accuracy. This superior performance stems from its ability to simultaneously capture level, slope, and curvature dynamics in mortality improvement while also incorporating cohort effects. For pension planning in Germany, we recommend either the Plat model or a model averaging approach that includes all four models to mitigate model risk and produce more robust forecasts.

The implications for the German pension system are significant: continued increases in life expectancy will place additional strain on the Pay-As-You-Go system, necessitating policy adjustments such as increases in the retirement age or modifications to the benefit formula. These findings underscore the importance of incorporating rigorous mortality forecasting into pension system design and reform.

\begin{thebibliography}{9}

\bibitem{leecarter1992}
Lee, R.D. and Carter, L.R. (1992).
\newblock Modeling and Forecasting U.S. Mortality.
\newblock \textit{Journal of the American Statistical Association}, 87(419):659--671.

\bibitem{cbd2006}
Cairns, A.J.G., Blake, D., and Dowd, K. (2006).
\newblock A Two-Factor Model for Stochastic Mortality with Parameter Uncertainty: Theory and Calibration.
\newblock \textit{Journal of Risk and Insurance}, 73(4):687--718.

\bibitem{apc2009}
Cairns, A.J.G., Blake, D., Dowd, K., Coughlan, G.D., Epstein, D., Ong, A., and Balevich, I. (2009).
\newblock A Quantitative Comparison of Stochastic Mortality Models Using Data from England and Wales and the United States.
\newblock \textit{North American Actuarial Journal}, 13(1):1--35.

\bibitem{destatis2023}
Statistisches Bundesamt (Destatis) (2023).
\newblock Sterbetafel (Periodensterbetafel): Deutschland, Jahre, Geschlecht, Vollendetes Alter.
\newblock Table 12621-0001. Available at: \url{https://www-genesis.destatis.de/}.

\bibitem{renshaw2006}
Renshaw, A.E. and Haberman, S. (2006).
\newblock A Cohort-Based Extension to the Lee-Carter Model for Mortality Reduction Factors.
\newblock \textit{Insurance: Mathematics and Economics}, 38(3):556--570.

\bibitem{plat2009}
Plat, R. (2009).
\newblock On Stochastic Mortality Modeling.
\newblock \textit{Insurance: Mathematics and Economics}, 45(3):393--404.

\end{thebibliography}

\end{document}
